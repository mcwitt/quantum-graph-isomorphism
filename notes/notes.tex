%        File: notes.tex
%     Created: Tue Jan 08 11:00 AM 2013 P
% Last Change: Tue Jan 08 11:00 AM 2013 P
%
\documentclass[a4paper]{article}
\usepackage{hyperref}

\newcommand{\ham}{\mathcal{H}}
\newcommand{\nn}[2]{\langle #1 #2 \rangle}
\newcommand{\av}[1]{\langle #1 \rangle}

\begin{document}
\begin{equation}
  \ham = (1 - s) \ham_D + s \ham_P,
  \qquad
  0 \leq s \leq 1
  \label{def-h_qaa}
\end{equation}
\begin{equation}
  \ham_D = \sum_i \sigma^x_i
  \label{def-h_d}
\end{equation}
\begin{equation}
  \ham_P = \sum_{\nn{i}{j}} \sigma^z_i \sigma^z_j + \sum_i h_i \sigma^z_i,
  \qquad
  h_i \rightarrow h
  \label{def-h_p}
\end{equation}
\begin{equation}
  F_{ij} = \av{ \sigma^z_i (h_j)}
  \label{def-f}
\end{equation}
Measure $F_{ij}(h_j \rightarrow h)$.
\begin{equation}
  \frac{\partial F_{ij}}{\partial h_j} (h_i \rightarrow h)
  \sim \av{\sigma^z_i \sigma^z_j}
  \label{}
\end{equation}
Can approximate derivative using finite differences. (What about Peter's
method?)

Numerical methods to consider:
\begin{enumerate}
  \item \href{http://en.wikipedia.org/wiki/Lanczos_algorithm}
    {Lanczos algorithm} (matrix diagonalization)
  \item \href{http://en.wikipedia.org/wiki/Conjugate_gradient_method}
    {Conjugate gradient method} (find ground state only)
  \item Power method
\end{enumerate}
\end{document}


